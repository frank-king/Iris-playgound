\subsection{Reasoning about Mutable Data Structures}
\setcounter{exercise}{18}

\begin{exercise}
  Explain why the separating conjunction ensures lists are acyclic. What goes
  wrong if we used ordinary conjunction?
  \begin{align*}
    \textlog{isList}\, l\, [] & \equiv l = \Inj_1\, () \\
    \textlog{isList}\, l\, (x :: xs) & \equiv \exists hd, l'.
    l = \Inj_2\,(hd) * hd \gmapsto (x, l') * \textlog{isList}\, l'\, xs
  \end{align*}
\end{exercise}

\begin{proof}
  By induction on $xs$, we show that the list $l$ is acyclic, for each node in the
  list, i.e. except its previous node (if present), there are no other nodes point
  to it.

  For the first case that $xs = []$, it is trivally acyclic.

  Then the second case where $xs = x :: xs'$ is interesting. We must show $l$ is
  acyclic with the induction hypothesis that the all but first node as a sublist
  $l'$ of $l$ is acyclic.

  The ``isList'' predicate is conjected by 3 separated propositions $l =
  \Inj_2(hd)$, $hd \gmapsto (x, l')$ and $\textlog{isList}\, l'\, xs'$.

  The first proposition tells that the list is not empty, the second proposition
  tells that it points to a head element $x$ and the remainder of the list $l'$,
  and the third proposition tells that the remainder of the list is also a list.

  From the second propsition we know that there cannot be another node in the list
  points to it because of the points-to predicate $\gmapsto$ is exclusive.

  From the third propsition and by the induciton hypothesis, we know that $l'$ is
  acyclic. Put them together, we are confident to claim that the whole list $l$ is
  acyclic.

  If we use ordinary conjunctions, we cannot guarantee that for any node in the
  list, there are no nodes other than its previous node point to it, so it might
  be cyclic.
\end{proof}

\setcounter{exercise}{19}
\begin{exercise}
  Let $\textlog{inc}$ denote the following program that increments all values in a
  linked list of integers:
  \begin{align*}
    \Rec \textlog{inc}(l)= \MatchML l with
    \Inj_1\, x_1=>()
    | \Inj_2\, x_2=>
    \begin{array}[t]{l}
      \Let h=\pi_1 \deref{x_2} in \\
      \Let t=\pi_2 \deref{x_2} in \\
      \mapelem{x_2}{(h + 1, t)}; \\
      \textlog{inc}\, t
    \end{array}
    end {}
  \end{align*}
  Prove this Hoare triple in detail using the rules just mentioned
  \begin{align*}
    \forall xs. \forall l. \hoare{\textlog{isList}\, l\, xs}{\textlog{inc}\, l}
    {v.v = () \wedge \textlog{isList}\, l\, (\textlog{map}\, (1+)\, xs)}
  \end{align*}
  where $\textlog{map}$ is defined by
  \begin{align*}
    \textlog{map}\, f & \equiv [] \\
    \textlog{map}\, (x :: xs) & \equiv f x :: \textlog{map}\, f\, xs
  \end{align*}
\end{exercise}

\begin{proof}
  Apply \rulenamestyle{Ht-rec},
  \begin{align}
    & \forall xs. \forall l. \hoare{\textlog{isList}\, l\, xs}{f\, l}
    {v.v = () \wedge \textlog{isList}\, l\, (\textlog{map}\, (1+)\, xs)} \proves
    \notag \\
    & \forall xs. \forall l. \hoare{\textlog{isList}\, l\, xs}
    {\MatchDots l with...end}
    {v.v = () \wedge \textlog{isList}\, l\, (\textlog{map}\, (1+)\, xs)}
    \label{exe4.20:1}
  \end{align}
  where the body of $\MatchDots l with...end$ has been substituted by $f$.

  There are two cases of $xs$, let's talk about them one by one.
  \begin{enumerate}
    \item $xs = []$, then $\textlog{isList}\, l\, [] \equiv l =
      \Inj_1\, ()$, and $\textlog{isList}\, l\,
      (\textlog{map}\, (1+)\, xs) \equiv \textlog{isList}\, l\, [] \equiv
      l = \Inj_1\, ()$. We have to show
      \begin{align}
        \forall l. \hoare{l = \Inj_1\, ()}{\MatchDots l with...end}
        {v.v = () \wedge l = \Inj_1\, ()}
        \tag{\ref{exe4.20:1}a} \label{4.20:1a}
      \end{align}
      Then apply \rulenamestyle{Ht-eq} and \rulenamestyle{Eq} to substitute $l$,
      \begin{align*}
        \hoare{\TRUE}{
          \Match \Inj_1\, () with
          \Inj_1\, x_1 => () | \Inj_2\, x_2 => \ldots end
        } {v.v = ()}
      \end{align*}

      Apply \rulenamestyle{Ht-match},
      \begin{align*}
        \hoare{\TRUE}{()} {v.v = ()}
      \end{align*}
      which is immediate from \rulenamestyle{Ht-ret}.

    \item $xs = x :: xs' $, then
      \begin{align*}
        \textlog{isList}\, l\, x :: xs' & \equiv
        \exists hd, l'. l = \Inj_2\,(hd) * hd \gmapsto (x, l') *
        \textlog{isList}\, l'\, xs', \\
        \textlog{isList}\, l\, (\textlog{map}\, (1+)\, xs) & \equiv
        \textlog{isList}\, l\, (1 + x) :: \textlog{map}\, (1+)\, xs' \\ & \equiv
        \exists hd, l'. l = \Inj_2\,(hd) * hd \gmapsto (1 + x, l') *
        \textlog{isList}\, l'\, \textlog{map}\, (1+)\, xs'.
      \end{align*}
      We have to show
      \begin{align}
        \forall xs. \forall l. & \hoare{\textlog{isList}\, l\, xs}{f\, l}
        {v.v = () \wedge \textlog{isList}\, l\, (\textlog{map}\, (1+)\, xs)} \proves
        \notag \\
        \forall l. & \hoareV[t]{\exists hd, l'. l = \Inj_2\,(hd) *
        hd \gmapsto (x, l') * \textlog{isList}\, l'\, xs'}
        {\MatchDots l with...end}
        {v.v = () \wedge \exists hd, l'. \left(
        \begin{array}[c]{l}
          l = \Inj_2\, (hd) * \\
          hd \gmapsto (1 + x, l') * \\
          \textlog{isList}\, l'\, \textlog{map}\, (1+)\, xs'
        \end{array}
        \right)}
        \tag{\ref{exe4.20:1}b} \label{4.20:1b}
      \end{align}

      Then apply \rulenamestyle{Ht-exist}, \rulenamestyle{Ht-eq} and
      \rulenamestyle{Eq} to substitute $l$,
      \begin{align*}
        \forall xs. \forall l. & \hoare{\textlog{isList}\, l\, xs}{f\, l}
        {v.v = () \wedge \textlog{isList}\, l\, (\textlog{map}\, (1+)\, xs)} \proves
        \notag \\
        \forall hd, l'. & \hoareV[t]{hd \gmapsto (x, l') *
        \textlog{isList}\, l'\, xs'}
        {\Match \Inj_2\, (hd) with
          \Inj_1\, x_1 => () | \Inj_2\, x_2 => \ldots end}
        {v.v = () \wedge hd \gmapsto (1 + x, l') *
        \textlog{isList}\, l'\, \textlog{map}\, (1+)\, xs'}
      \end{align*}


      Apply \rulenamestyle{Ht-match},
      \begin{align*}
        \forall xs. \forall l. & \hoare{\textlog{isList}\, l\, xs}{f\, l}
        {v.v = () \wedge \textlog{isList}\, l\, (\textlog{map}\, (1+)\, xs)} \proves
        \notag \\
        \forall hd, l'. & \hoareV[t]{hd \gmapsto (x, l') *
        \textlog{isList}\, l'\, xs'}
        {
          \begin{array}[c]{l}
            \Let h=\pi_1 \deref{hd} in \\
            \Let t=\pi_2 \deref{hd} in \\
            \mapelem{x_2}{(h + 1, t)}; \\
            f\, t
          \end{array}
        }
        {v.v = () \wedge hd \gmapsto (1 + x, l') *
        \textlog{isList}\, l'\, \textlog{map}\, (1+)\, xs'}
      \end{align*}

      Apply \rulenamestyle{Ht-let-det} twice,
      \begin{align*}
        \forall xs. \forall l. & \hoare{\textlog{isList}\, l\, xs}{f\, l}
        {v.v = () \wedge \textlog{isList}\, l\, (\textlog{map}\, (1+)\, xs)} \proves
        \notag \\
        \forall hd, l'. & \hoareV[t]{hd \gmapsto (x, l') *
        \textlog{isList}\, l'\, xs'}
        {\mapelem{x_2}{(x + 1, l')}; f\, l'}
        {v.v = () \wedge hd \gmapsto (1 + x, l') *
        \textlog{isList}\, l'\, \textlog{map}\, (1+)\, xs'}
      \end{align*}

      With no doubt, it can be derived by applying \rulenamestyle{Ht-load}
      and \rulenamestyle{Ht-store}.
  \end{enumerate}
\end{proof}

\setcounter{exercise}{22}
\begin{exercise}
  Derive the rule \rulenamestyle{Ht-Rec-multi}.
  \begin{mathpar}
    \infer[Ht-Rec-multi] {
      \vctx, g: \Val \mid S \wedge \forall z. \forall v_1. \forall v_2.
      \hoare{P}{g(v_1 , v_2)}{u.Q} \proves \forall z. \forall v_1 . \forall v_2.
      \hoare{P}{\subst{\subst{\subst{e}{f}{g}}{x}{v_1}}{y}{v_2}}{u.Q}
    } {
      \vctx \mid S \proves \forall z. \forall v.
      \hoare{\exists v_1 , v_2. v = (v_1, v_2) \wedge P}{(\Rec f(x, y)=e) v}
      {u. \exists v_1, v_2. v = (v_1, v_2) \wedge Q}
    }
  \end{mathpar}
  where $\Rec f(x, y)=e$ is the syntax sugar of
  \begin{align*}
    \Rec f(p)=\Let x=\pi_1\, p in \Let y=\pi_2\, in e.
  \end{align*}
  and we assume that the variable $v$ is fresh for $P$ and $Q$
\end{exercise}

\begin{proof}
  Apply \rulenamestyle{Ht-exist}, \rulenamestyle{Ht-rec} and
  \rulenamestyle{$\forall$I},
  \begin{align*}
    \vctx, g, v: \Val \mid S \wedge \forall z. \forall v_1. \forall v_2
    & \hoare{v = (v_1, v_2) \wedge P}{g\, v}
    {u.\exists v_1, v_2. v = (v_1, v_2) \wedge Q} \proves \\
    \forall z. \forall v_1. \forall v_2
    & \hoareV[t]{v = (v_1, v_2) \wedge P}
    {\Let x=\pi_1\, v in \Let y=\pi_2\, v in \subst{e}{f}{g}}
    {u. \exists v_1, v_2. v = (v_1, v_2) \wedge Q}
  \end{align*}

  Apply \rulenamestyle{Ht-Ht}, \rulenamestyle{Ht-eq-pre}, and \rulenamestyle{Ht-Ht}
  again, then omit the trivial $\exists v_1', v_2'. (v_1, v_2) = (v_1', v_2')$,
  \begin{align*}
    \vctx, g: \Val \mid S \wedge \forall z. \forall v_1. \forall v_2
    & \hoare{P}{g\, (v_1, v_2)} {u.Q} \proves \\
    \forall z. \forall v_1. \forall v_2 & \hoare{P}
    {\Let x=\pi_1\, (v_1, v_2) in \Let y=\pi_2\, (v_1, v_2) in \subst{e}{f}{g}}
    {u.Q}
  \end{align*}
  From now on, we omit the context and asumption of the above entailment,
  since they are the same as those in the hypothesis.

  Apply \rulenamestyle{Ht-let-det}, \rulenamestyle{Ht-proj} and
  \rulenamestyle{Ht-frame},
  \begin{align*}
    \forall z. \forall v_1. \forall v_2 & \hoare{P}
    {\Let y=\pi_2\, (v_1, v_2) in \subst{\subst{e}{f}{g}}{x}{v_1}}{u.Q}
  \end{align*}
  Again,
  \begin{align*}
    \forall z. \forall v_1. \forall v_2 & \hoare{P}
    {\subst{\subst{\subst{e}{f}{g}}{x}{v_1}}{y}{v_2}}{u.Q}
  \end{align*}
  which is immediate from the hypothsis.
\end{proof}

\begin{exercise}
  Let $\textlog{append}$ be the following function, which takes two linked lists
  as arguments and returns a list which is the concatenation of the two.
  \begin{align*}
    \Rec \textlog{append}(l,l')=\MatchML l with
    \Inj_1\, x_1=>l'
    | \Inj_2\, x_2=>
    \begin{array}[t]{l}
      \Let p=\deref{x_2} in \\
      \Let r=\textlog{append}\, (\pi_2\, p)\, l' in \\
      \mapelem{x_2}{(\pi_1\, p, r)}; \\
      \Inj_2\, x_2
    \end{array}
    end {}
  \end{align*}
  We wish to give it the following specification where $\dplus$ is append on
  mathematical sequences.
  \begin{align*}
    \forall xs, ys, l, l'.
    \hoare{\textlog{isList}\, l\, xs * \textlog{isList}\, l' ys}
    {\textlog{append}\, l\, l'}{v. \textlog{isList}\, v\, (xs \dplus ys)}.
  \end{align*}
  \begin{itemize}
    \item Prove the specification.
    \item Is the following specification also valid?
      \begin{align*}
        \forall xs, ys, l, l'.
        \hoare{\textlog{isList}\, l\, xs \wedge \textlog{isList}\, l'\, ys}
        {\textlog{append}\, l\, l'}{v. \textlog{isList}\, v\, (xs \dplus ys)}
      \end{align*}
      Hint: Think about what is the result of $\textlog{append}\, l\, l'$.
  \end{itemize}
\end{exercise}

\begin{proof}
  \begin{itemize}
    \item It is trivial to derive this rule from \rulenamestyle{Ht-rec-multi},
      \begin{align*}
        \infer[Ht-Rec-multi$'$] {
          \vctx, g: \Val \mid S \wedge \forall z. \forall v_1. \forall v_2.
          \hoare{P}{g(v_1 , v_2)}{u.Q} \proves \forall z. \forall v_1 . \forall v_2.
          \hoare{P}{\subst{\subst{\subst{e}{f}{g}}{x}{v_1}}{y}{v_2}}{u.Q}
        } {
          \vctx \mid S \proves \forall z. \forall v_1. \forall v_2.
          \hoare{P}{(\Rec f(x, y)=e)\, v_1\, v_2}{u. Q}
        }
      \end{align*}
      Then apply it to the goal,
      \begin{align*}
        \vctx, g: \Val \mid \forall xs, ys, l, l'. &
        \hoare{\textlog{isList}\, l\, xs * \textlog{isList}\, l' ys}
        {g\, l\, l'}{v. \textlog{isList}\, v\, (xs \dplus ys)} \\
        \proves \forall xs, ys, l, l'. &
        \hoareV[t]{\textlog{isList}\, l\, xs * \textlog{isList}\, l' ys} {
          \MatchML l with
          \Inj_1\, x_1=>l'
          | \Inj_2\, x_2=>
          \begin{array}[t]{l}
            \Let p=\deref{x_2} in \\
            \Let r=g\, (\pi_2\, p)\, l' in \\
            \mapelem{x_2}{(\pi_1\, p, r)}; \\
            \Inj_2\, x_2
          \end{array}
          end {}
        } {v. \textlog{isList}\, v\, (xs \dplus ys)}
      \end{align*}
      We omit the assumption to reduce verboseness, and consider the two cases
      of $xs$ in $\textlog{isList}$.
      \begin{align}
        & \forall ys, l, l'.
        \hoare{l = \Inj_1\, () * \textlog{isList}\, l' ys}
        {\MatchDots l with...end} {v. \textlog{isList}\, v\, ys}
        \label{exe4.24:1} \\
        & \forall x, xs', hd, l_1, ys, l, l'.
        \hoareV[t]
        {l = \Inj_2\, (hd) * hd \gmapsto (x, l_1) *
        \textlog{isList}\, l_1\, xs' * \textlog{isList}\, l' ys}
        {\MatchDots l with...end} {v. \textlog{isList}\, v\, (x :: xs' \dplus ys)}
        \label{exe4.24:2}
      \end{align}
      Apply \rulenamestyle{$\forall$I} and \rulenamestyle{Ht-eq-pre} to eliminate
      $l$,
      \begin{align}
        & \forall ys, l'.  \hoare{\textlog{isList}\, l' ys}
        {\MatchDots \Inj_1\, ()  with...end}
        {v. \textlog{isList}\, v\, ys}
        \tag{\ref{exe4.24:1}a} \label{exe4.24:1a} \\
        & \forall x, xs', hd, l_1, ys, l'.
        \hoareV[t]{hd \gmapsto (x, l_1) *
        \textlog{isList}\, l_1\, xs' * \textlog{isList}\, l' ys}
        {\MatchDots \Inj_2\, (hd)  with...end}
        {v. \textlog{isList}\, v\, (x :: xs' \dplus ys)}
        \tag{\ref{exe4.24:2}a} \label{exe4.24:2a}
      \end{align}

      Then apply \rulenamestyle{Ht-match} to extract the arms in $\langkw{match}$,
      \begin{align}
        & \forall ys, l'.
        \hoare{\textlog{isList}\, l' ys}{l'}{v. \textlog{isList}\, v\, ys}
        \tag{\ref{exe4.24:1}b} \label{exe4.24:1b} \\
        & \forall x, xs', hd, l_1, ys, l'.\
        \hoareV[t]{hd \gmapsto (x, l_1) *
        \textlog{isList}\, l_1\, xs' * \textlog{isList}\, l' ys} {
          \begin{array}[t]{l}
            \Let p=\deref{hd} in \\
            \Let r=g\, (\pi_2\, p)\, l' in \\
            \mapelem{hd}{(\pi_1\, p, r)}; \\
            \Inj_2\, hd
          \end{array}
        } {v. \textlog{isList}\, v\, (x :: xs' \dplus ys)}
        \tag{\ref{exe4.24:2}b} \label{exe4.24:2b}
      \end{align}

      \eqref{exe4.24:1b} is obvious by applying \rulenamestyle{Ht-frame},
      \rulenamestyle{Ht-ret} and the induction hypothesis.

      Apply \rulenamestyle{Ht-let-det}, \rulenamestyle{Ht-load} and
      \rulenamestyle{Ht-eq-ref} on \eqref{exe4.24:2b} to eliminate $p$,
      \begin{align*}
        \forall x, xs', hd, l_1, ys, l'.
        \hoareV[t]{hd \gmapsto (x, l_1) *
        \textlog{isList}\, l_1\, xs' * \textlog{isList}\, l' ys} {
          \begin{array}[t]{l}
            \Let r=g\, (\pi_2\, (x, l_1))\, l' in \\
            \mapelem{hd}{(\pi_1\, (x, l_1), r)}; \\
            \Inj_2\, hd
          \end{array}
        } {v. \textlog{isList}\, v\, (x :: xs' \dplus ys)}
      \end{align*}

      Again, apply \rulenamestyle{Ht-let-det}, \rulenamestyle{Ht-bind-det},
      \rulenamestyle{Ht-proj}, \rulenamestyle{Ht-eq-ref} to eliminate $r$,
      \begin{align*}
        \forall x, xs', hd, l_1, ys, l'.
        \hoareV[t]{hd \gmapsto (x, l_1) *
        \textlog{isList}\, l_1\, xs' * \textlog{isList}\, l' ys}
        {\mapelem{hd}{(\pi_1\, (x, l_1), (g\, l_1\, l'))}; \Inj_2\, hd}
        {v. \textlog{isList}\, v\, (x :: xs' \dplus ys)}
      \end{align*}

      Apply \rulenamestyle{Ht-bind-det} and \rulenamestyle{Ht-proj} to eliminate
      $\pi_1\, (x, l_1)$, and apply \rulenamestyle{Ht-bind} to eliminate $g\, l_1,
      l'$. Note that we cannot elimitate $g$ immediately because we don't know the
      value after executing it.
      \begin{align}
        \forall x, xs', hd, l_1, ys, l'. &
        \hoare{hd \gmapsto (x, l_1) *
        \textlog{isList}\, l_1\, xs' * \textlog{isList}\, l' ys}
        {g\, l_1\, l'}{u. P}
        \tag{\ref{exe4.24:2}c} \label{exe4.24:2c} \\
        \forall x, xs', hd, l_1, ys, u. &
        \hoare{P}{\mapelem{hd}{(x, u)}; \Inj_2\, hd}
        {v. \textlog{isList}\, v\, (x :: xs' \dplus ys)}
        \tag{\ref{exe4.24:2}d} \label{exe4.24:2d}
      \end{align}

      By the induction hypothesis and \rulenamestyle{Ht-frame}, we know that
      \eqref{exe4.24:2c} holds for $P \equiv hd \gmapsto (x, l_1) *
      \textlog{isList}\, u\, (xs' \dplus ys)$. So \eqref{exe4.24:2d} becomes
      \begin{align*}
        \forall x, xs', hd, l_1, ys, u. &
        \hoareV[t]{hd \gmapsto (x, l_1) * \textlog{isList}\, u\, (xs' \dplus ys)}
        {\mapelem{hd}{(x, u)}; \Inj_2\, hd}
        {v. \textlog{isList}\, v\, (x :: xs' \dplus ys)}
      \end{align*}

      Apply \rulenamestyle{Ht-store} and \rulenamestyle{Ht-Seq},
      \begin{align*}
        \forall x, xs', hd, l_1, ys, v. &
        \hoareV[t]{hd \gmapsto (x, u) * \textlog{isList}\, u\, (xs' \dplus ys)}
        {\Inj_2\, hd}
        {v. v = \Inj_2\, hd \wedge
        \textlog{isList}\, v\, (x :: xs' \dplus ys)}
      \end{align*}
      Expand $\textlog{isList}\, v\, (x :: xs' \dplus ys)$, then it becomes
      \begin{align*}
        \forall x, xs', hd, l_1, ys, v. &
        \hoareV[t]{hd \gmapsto (x, u) * \textlog{isList}\, u\, (xs' \dplus ys)}
        {\Inj_2\, hd} {
          v. v = \Inj_2\, hd \wedge \exists hd', l_1'.
          \begin{array}[c]{l}
            v = \Inj_2 (hd') * \\
            hd' \gmapsto (x, l_1') * \\
            \textlog{isList}\, l_1'\, (xs' \dplus ys)
          \end{array}
        }
      \end{align*}
      Apparently, $\exists hd, u$ such that
      \begin{align*}
        \forall x, xs', hd, l_1, ys, v. &
        \hoareV[t]{hd \gmapsto (x, u) * \textlog{isList}\, u\, (xs' \dplus ys)}
        {\Inj_2\, hd}
        {v. v = \Inj_2\, hd * hd \gmapsto (x, u) *
        \textlog{isList}\, u\, (xs' \dplus ys)}
      \end{align*}
      which is immediately by \rulenamestyle{Ht-frame} and \rulenamestyle{Ht-ret}.

    \item It is not valid. Because the ordinary conjuntion cannot guarantee that
      list $l$ and $l'$ don't overlap with each other. When there are overlaps,
      $\textlog{append}\, l\, l'$ can construct a cyclic list.
  \end{itemize}
\end{proof}

\begin{exercise}
  The $\textlog{append}$ function in the previous exercise is not tail recursive
  and hence it space consumption is linear in the length of the first list. A
  better implementation of append for linked lists is the following.
  \begin{align*}
    \textlog{append}’\, l\, l'= \spac & \Let go=\Rec f(h\, p)=
    \MatchML h with
    \Inj_1\, x_1=>\mapelem{p}{(\pi_1\, (\deref{p}), l')}
    | \Inj_2\, x_2=>f\, (\pi_2\, (\deref{x_2}))\, x_2
    end {} \\
    & in \MatchML l with
    \Inj_1\, x_1=>l'
    | \Inj_2\, x_2=>go\, (\pi_2 (\deref{x_2}))\, x_2;\; l
    end {}
  \end{align*}
  In the function $go$ the value $p$ is the last node of the list $l$ we have seen
  while traversing $l$. Thus $go$ traverses the first list and once it reaches the
  end it updates the tail pointer in the last node to point to the second list,
  $l'$ .

  Prove for $\textlog{append}’$ the same specification as for $\textlog{append}$
  above. You need to come up with a strong enough invariant for the function $go$,
  relating $h$, $p$ and $xs$ and $ys$.
\end{exercise}

\begin{proof}
  We are going to prove
  \begin{align}
    \forall xs, ys, l, l'.
    \hoare{\textlog{isList}\, l\, xs * \textlog{isList}\, l' ys}
    {\textlog{append}'\, l\, l'}{v. \textlog{isList}\, v\, (xs \dplus ys)}.
    \label{exe4.25:1}
  \end{align}

  First, consider this recursive function
  \begin{align*}
    \Rec f(h\, p)=
    \MatchML h with
    \Inj_1\, x_1=>\mapelem{p}{(\pi_1\, (\deref{p}), l')}
    | \Inj_2\, x_2=>f\, (\pi_2\, (\deref{x_2}))\, x_2
    end {}
  \end{align*}

  From the first arm of $\langkw{match}$ we know that $l'$ is appended after at
  $p$ which is a pointer of a node, at the end of the function.

  From the second arm of $\langkw{match}$ we know that $h$ is the ``next''
  pointer of $p$, which is optional, and after an iteration, both $h$ and $p$ move
  to the next node.

  Based on these observations, we construct the following invariant of $f$ and
  prove it.
  \begin{align}
    \forall l', x, h, p, xs, ys.
    \hoareV[t]{p \gmapsto (x, h) * \textlog{isList}\, h\, xs *
    \textlog{isList}\, l'\, ys}
    {f\, h\, p}
    {v.v = () \wedge \exists h. p \gmapsto (x, h).
    \textlog{isList}\, h\, (xs \dplus ys)}
    \label{exe4.25:2}
  \end{align}
  Here $h$ might change after execution of $f\, h\, p$, but $p$ always remains
  unchanged.

  Apply \rulenamestyle{Ht-rec}, and we have the induction hypothesis
  \label{exe4.26:2} and the new goal \label{exe4.26:3}
  \begin{align}
    \vctx, g: \Val \mid \forall l', x, h, p, xs, ys.
    & \hoareV[t]{p \gmapsto (x, h) * \textlog{isList}\, h\, xs *
    \textlog{isList}\, l'\, ys}
    {g\, h\, p}
    {v.v = () \wedge \exists h. p \gmapsto (x, h) *
    \textlog{isList}\, h\, (xs \dplus ys)}
    \label{exe4.25:3} \\
    \proves \forall l', x, h, p, xs, ys.
    & \hoareV[t]{p \gmapsto (x, h) * \textlog{isList}\, h\, xs *
    \textlog{isList}\, l'\, ys} {
      \MatchML h with
      \Inj_1\, x_1=>\mapelem{p}{(\pi_1\, (\deref{p}), l')}
      | \Inj_2\, x_2=>g\, (\pi_2\, (\deref{x_2}))\, x_2
      end {}
    } {v.v = () \wedge \exists h. p \gmapsto (x, h) *
    \textlog{isList}\, h\, (xs \dplus ys)}
    \label{exe4.25:4}
  \end{align}

  Consider 2 cases of $xs$ and apply \rulenamestyle{Ht-eq-pre},
  \rulenamestyle{Ht-match} and \rulenamestyle{Ht-exist} on \eqref{exe4.25:4},
  \begin{align}
    \forall l', x, p, ys.
    & \hoareV[t]{p \gmapsto (x, \Inj_1\, ()) * \textlog{isList}\, l'\, ys}
    {\mapelem{p}{(\pi_1\, (\deref{p}), l')}}
    {v.v = () \wedge \exists h. p \gmapsto (x, h) * \textlog{isList}\, h\, ys}
    \tag{\ref{exe4.25:4}a} \label{exe4.25:4a} \\
    \forall l', x, x', hd, h', p, xs', ys.
    & \hoareV[t]{
      \begin{array}[c]{l}
        p \gmapsto (x, \Inj_2\, (hd)) * \\
        hd \gmapsto (x', h') * \\
        \textlog{isList}\, h'\, xs' * \\
        \textlog{isList}\, l'\, ys
      \end{array}
    } {g\, (\pi_2\, (\deref{hd}))\, hd} {
      v.v = () \wedge \exists h.
        p \gmapsto (x, h) *
        \textlog{isList}\, h\, (x' :: xs' \dplus ys)
    }
    \tag{\ref{exe4.25:4}b} \label{exe4.25:4b}
  \end{align}

  \eqref{exe4.25:4a} can be obtained by applying \rulenamestyle{Ht-bind-det},
  \rulenamestyle{Ht-load}, \rulenamestyle{Ht-bind-det}, \rulenamestyle{Ht-proj},
  and \rulenamestyle{Ht-store}, where the last step is
  \begin{align*}
    \forall l', x, p, ys.
    \hoareV[t]{p \gmapsto (x, \Inj_1\, ()) * \textlog{isList}\, l'\, ys}
    {\mapelem{p}{x, l')}}
    {v.v = () \wedge p \gmapsto (x, l') * \textlog{isList}\, l'\, ys}
  \end{align*}
  which is immediate from \rulenamestyle{Ht-store}.

  Similarly, apply \rulenamestyle{Ht-bind-det}, \rulenamestyle{Ht-load},
  \rulenamestyle{Ht-bind-det}, and \rulenamestyle{Ht-proj} to eliminate $hd$
  in it,
  \begin{align*}
    & \forall l', x, x', hd, h', p, xs', ys. \\
    & \hoareV[t]{
        p \gmapsto (x, \Inj_2\, (hd)) * hd \gmapsto (x', h') *
        \textlog{isList}\, h'\, xs' * \textlog{isList}\, l'\, ys
    } {g\, h'\, hd} {
      v.v = () \wedge \exists h. p \gmapsto (x, h) *
        \textlog{isList}\, h\, (x' :: xs' \dplus ys)
    }
  \end{align*}
  Let $h \equiv \Inj_2\, (hd)$ and $xs \equiv x' :: xs'$, and fold the first
  $\textlog{isList}$, then we have
  \begin{align*}
    \forall l', x, h, p, xs, ys.
    \hoareV[t]
    {p \gmapsto (x, h) * \textlog{isList}\, h\, xs * \textlog{isList}\, l'\, ys}
    {g\, h'\, hd}
    {v.v = () \wedge \exists h. p \gmapsto (x, h) *
        \textlog{isList}\, h\, (x' :: xs' \dplus ys)}
  \end{align*}

  It is immediate from the induction hypothesis \eqref{exe4.25:3}.

  Now we can use \eqref{exe4.25:2} to prove \eqref{exe4.25:1}. First apply
  \rulenamestyle{Ht-let-det} to eliminate $go$,
  \begin{align}
    \forall xs, ys, l, l'.
    \hoareV[t]{\textlog{isList}\, l\, xs * \textlog{isList}\, l' ys} {
      \MatchML l with
      \Inj_1\, x_1=>l'
      | \Inj_2\, x_2=>(\Rec f(h\, p)=\ldots)\, (\pi_2\, \deref{(x_2)})\, x_2;\; l
      end {}
    } {v. \textlog{isList}\, v\, (xs \dplus ys)}
    \label{exe4.25:5}
  \end{align}

  Consider 2 cases of $xs$, and apply \rulenamestyle{Ht-eq-pre},
  \rulenamestyle{Ht-match}, and \rulenamestyle{Ht-exist} on \eqref{exe4.25:5}
  \begin{align}
    \forall ys, l, l'.
    & \hoare{\textlog{isList}\, l' ys}{l'}{v. \textlog{isList}\, v\, ys}
    \tag{\ref{exe4.25:5}a} \label{exe4.25:5a} \\
    \forall x, xs', ys, hd, l_1, l'.
    & \hoareV[t]{hd \gmapsto (x, l_1) *
    \textlog{isList}\, l_1\, xs' * \textlog{isList}\, l' ys}
    {(\Rec f(h\, p)=\ldots)\, (\pi_2\, \deref{(hd)})\, hd;\; \Inj_2\, hd}
    {v. \textlog{isList}\, v\, (x :: xs' \dplus ys)}
    \tag{\ref{exe4.25:5}b} \label{exe4.25:5b}
  \end{align}
  \eqref{exe4.25:5a} is immediate by \rulenamestyle{Ht-ret}.

  Apply \rulenamestyle{Ht-bind-det}, \rulenamestyle{Ht-load},
  \rulenamestyle{Ht-bind-det}, and \rulenamestyle{Ht-proj} to eliminate
  $\pi_2\, \deref{(hd)}$
  \begin{align}
    \forall x, xs', ys, hd, l_1, l'.
    & \hoareV[t]{hd \gmapsto (x, l_1) *
    \textlog{isList}\, l_1\, xs' * \textlog{isList}\, l' ys}
    {(\Rec f(h\, p)=\ldots)\, l_1\, hd;\; \Inj_2\, hd}
    {v. \textlog{isList}\, v\, (x :: xs' \dplus ys)}
    \tag{\ref{exe4.25:5}c} \label{exe4.25:5c}
  \end{align}

  From \eqref{exe4.25:2} we know that
  \begin{align}
    \forall x, xs', ys, hd, l_1, l'.
    & \hoareV[t]{hd \gmapsto (x, l_1) *
    \textlog{isList}\, l_1\, xs' * \textlog{isList}\, l' ys}
    {(\Rec f(h\, p)=\ldots)\, l_1\, hd}
    {v. v = () \wedge \exists h. hd \gmapsto (x, h) *
    \textlog{isList}\, h\, (xs' \dplus ys)}
    \tag{\ref{exe4.25:5}d} \label{exe4.25:5d}
  \end{align}

  Then apply \rulenamestyle{Ht-seq} to \eqref{exe4.25:5c} and associate it
  with \eqref{exe4.25:5d}
  \begin{align*}
    \forall x, xs', ys, hd, l_1, l'.
    & \hoareV[t]{\exists h. hd \gmapsto (x, h) *
    \textlog{isList}\, h\, xs' * \textlog{isList}\, l' ys}
    {\Inj_2\, hd}
    {v. \textlog{isList}\, v\, (x :: xs' \dplus ys)}
  \end{align*}

  It can be derived by \rulenamestyle{Ht-exist}, \rulenamestyle{Ht-ret},
  and the definition of $\textlog{isList}$.
\end{proof}

Exercises 4.26 -- 4.30 are done in Coq.
