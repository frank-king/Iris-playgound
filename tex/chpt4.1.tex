\subsection{Derived rules for Hoare triples}
\begin{exercise}
  Show the following derived rule
  \begin{mathpar}
    \infer[Ht-Pre-Eq] {
      \vctx\mid S \proves 
      \hoare{\subst{P}{x}{v}}{\subst{e}{x}{v}}{u.\subst{Q}{x}{v}}
    } {\vctx, x: \Val \mid S \proves \hoare{x = v \wedge P}{e}{u.Q}}
  \end{mathpar}
\end{exercise}

\begin{proof}
  \begin{mathpar}
    \infern{Ht-Eq} {
      \inferN{Eq} {
        \inferrule*{
          \vctx\mid S \proves 
          \hoare{\subst{P}{x}{v}}{\subst{e}{x}{v}}{u.\subst{Q}{x}{v}}
        } {
          \vctx, x: \Val \mid S \wedge x = v \proves 
          \hoare{\subst{P}{x}{v}}{\subst{e}{x}{v}}{u.\subst{Q}{x}{v}}
        } \and
        S \wedge x = v \proves x = v
      } {\vctx, x: \Val \mid S \wedge x = v \proves \hoare{P}{e}{u.Q}}
    } {\vctx, x: \Val \mid S \proves \hoare{x = v \wedge P}{e}{u.Q}}
  \end{mathpar}
\end{proof}

\begin{exercise}
  Recall we define the let expression $\Let x = e1 in e2$ using abstraction and 
  application. Show the following derived rule
  \begin{mathpar}
    \infer[Ht-let] {
      S \proves \hoare{P}{e_1}{x.Q} \and
      S \proves \forall v. \hoare{\subst{Q}{x}{v}}{\subst{e_2}{x}{v}}{u.R}
    } {S \proves \hoare{P}{\Let x = e_1 in e_2}{u.R}}
  \end{mathpar}
  Using this rule is perhaps a bit inconvenient since most of the time the result
  of evaluating $e_1$ will be a single value and the postcondition $Q$ will be of
  the form $x = v \wedge Q'$ for some value $v$.

  The following rule, a special case of the above rule reflects this common case.
  Derive the rule from the rule \rulenamestyle{Ht-let}.
  \begin{mathpar}
    \infer[Ht-let-det] {
      S \proves \hoare{P}{e_1}{x.x = v \wedge Q} \and
      S \proves \hoare{\subst{Q}{x}{v}}{\subst{e_2}{x}{v}}{u.R}
    } {S \proves \hoare{P}{\Let x = e_1 in e_2}{u.R}}
  \end{mathpar}
  Define the sequencing expression $e_1; e_2$ such that when this expression is
  run first $e_1$ is evaluated to a value, the value is discarded, and then $e_2$
  is evaluated. Show the following specifications for the defined construct.
  \begin{mathpar}
    \infer[Ht-seq] {
      S \proves \hoare{P}{e_1}{v.Q} \and
      S \proves \hoare{\exists x. Q}{e_2}{u.R}
    } {
      S \proves \hoare{P}{e_1; e_2}{u.R}
    } \and
    \infer {
      S \proves \hoare{P}{e_1}{\_.Q} \and
      S \proves \hoare{Q}{e_2}{u.R}
    } {
      S \proves \hoare{P}{e_1 ; e_2}{u.R}
    }
  \end{mathpar}
  where $\_.Q$ means that $Q$ does not mention the return value.
\end{exercise}

\begin{proof}
  \begin{enumerate}
    \item Notice that $\Let x = e_1 in e_2$ is desugared to $(\Rec f(x)=e_2) e_1$,
      then \rulenamestyle{Ht-let} becomes
      \begin{mathpar}
        \infer {
          S \proves \hoare{P}{e_1}{x.Q} \and
          S \proves \forall v. \hoare{\subst{Q}{x}{v}}{\subst{e_2}{x}{v}}{u.R}
        } {S \proves \hoare{P}{(\Rec f(x)=e_2) e_1}{u.R}}
      \end{mathpar}

      Applying \rulenamestyle{Ht-bind} to $S \proves 
      \hoare{P}{(\Rec f(x)=e_2) e_1}{u.R}$, we have to show
      \begin{align}
        S \proves & \hoare{P}{e_1}{x.Q}, \label{exe4.12:1} \\
        S \proves & \forall v. \hoare{\subst{Q}{v}{x}}{(\Rec f(x)=e_2) v}{u.R},
        \label{exe4.12:2}
      \end{align}
      where the original $x$ \eqref{exe4.12:2} has been substituted by $v$ in
      order not to be confused with the $x$ in $f(x)$.

      \eqref{exe4.12:1} is immediate from the first hypothsis. Applying
      \rulenamestyle{Ht-rec} to \eqref{exe4.12:2}, we have to show
      \begin{align*}
        \vctx, g: \Val \mid S \wedge \forall v. \hoare{Q}{g\, v}{u.R} \proves
        \forall v. \hoare{\subst{Q}{x}{v}}{\subst{\subst{e_2}{f}{g}}{x}{v}}{u.R}
      \end{align*}

      Since $\Rec f(x)=e_2$ is not recursive, $e_2$ does not contain $f$,
      we can omit the $g$, then
      \begin{align*}
        \vctx, g: \Val \mid S \wedge \forall v. \hoare{Q}{g\, v}{u.R} \proves
        \forall v. \hoare{\subst{Q}{x}{v}}{\subst{e_2}{x}{v}}{u.R}
      \end{align*}
      which is immediate from the second hypothsis.
      
    \item Notice that $e_1; e_2$ is desugared to $\Let \_ = e_1 in e_2$, then
      \rulenamestyle{Ht-seq} is immediate from \rulenamestyle{Ht-let-det}.
  \end{enumerate}
\end{proof}

\begin{exercise}
  Recall we defined $\Lam x.e$ to be $\Rec f(x)=e$ where $f$ is some fresh variable.
  Show the following derived rule.
  \begin{mathpar}
    \infer[Ht-beta] 
    {S \proves \hoare{P}{\subst{e}{x}{v}}{u.Q}}
    {S \proves \hoare{P}{(\Lam x.e)v}{u.Q}}
  \end{mathpar}
\end{exercise}

\begin{proof}
  Desugar $(\Lam x.e) v$ to $(\Rec f(x)=e) v$, and apply \rulenamestyle{Ht-rec}
  to it, then we have to show
  \begin{align*}
    \vctx, g: \Val \mid S \wedge \forall v. \hoare{P}{g\, v}{u.Q} \proves
    \hoare{P}{\subst{e}{x}{v}}{u.Q}
  \end{align*}
  which is immediate from the hypothesis.
\end{proof}

\begin{exercise}
  Derive the following rule.
  \begin{mathpar}
    \infer[Ht-bind-det] {
      E \text{is an eval. context} \and
      S \proves \hoare{P}{e}{x.x = u \wedge Q} \and
      S \proves \hoare{\subst{Q}{x}{u}}{E[u]}{w. R}
    } {S \proves \hoare{P}{E[e]}{w. R}}
  \end{mathpar}
\end{exercise}

\begin{proof}
  Apply \rulenamestyle{Ht-bind} with $Q := x = u \wedge Q$ to the goal, then we
  have to show
  \begin{align*}
    S \proves \forall v. \hoare{u = v \wedge \subst{Q}{x}{v}}{E[v]}{w. R}
  \end{align*}
  
  Extract $u = v$ with \rulenamestyle{Ht-eq} and substitute $v$ with $u$ in the
  above entailment,
  \begin{align*}
    S \wedge u = u \proves \hoare{\subst{Q}{x}{u}}{E[u]}{w. R}
  \end{align*}
  which is immediate from the hypothesis.
\end{proof}

\setcounter{exercise}{15}
\begin{exercise}
  Show the following triples and entailments in detail.
  \begin{enumerate}
    \item \[ \hoare{R * \loc \gmapsto m}{\mapelem{\loc}{\deref{\loc} + 5}}
      {v.R * v = () * \loc \gmapsto (m + 5)} \]
    \item \[ \hoare{P}{\subst{e}{x}{m + 5}}{v.Q} \proves
      \hoare{P * \loc \gmapsto m}{\Let x = \deref{\loc} + 5 in e}
      {v.Q * \loc \gmapsto m} \]
    \item Assuming $u$ does not appear in $P$ and $Q$ show the following
      entailment.
      \[ \hoare{P}{\subst{e}{x}{v_1}}{v.Q} \proves
      \hoare{u = (v_1, v_2) * P}{\Let x = \pi_1 u in e}{v.Q} \]
  \end{enumerate}
\end{exercise}

\begin{proof}
  \begin{enumerate}
    \item \label{exe4.16.1} 
      Apply \rulenamestyle{Ht-bind-det} to the goal
      \begin{align}
        \hoare{\loc \gmapsto m}{\mapelem{\loc}{\deref{\loc} + 5}}
        {v.v = () * \loc \gmapsto (m + 5)} \label{exe4.16:1}
      \end{align}
      
      Apply \rulenamestyle{Ht-bind-det} to \eqref{exe4.16:1},
      \begin{align}
        & \hoare{\loc \gmapsto m}{\deref{\loc} + 5}{v.v = v_1 * Q_1},
        \label{exe4.16:2} \\
        & \hoare{\subst{Q_1}{v}{v_1}}{\mapelem{\loc}{v_1}}
        {v.v = () * \loc \gmapsto (m + 5)} \label{exe4.16:3}
      \end{align}

      Apply \rulenamestyle{Ht-bind-det} to \eqref{exe4.16:2},
      \begin{align}
        & \hoare{\loc \gmapsto m}{\deref{\loc}}{v.v = v_2 * Q_2},
        \tag{\ref{exe4.16:2}a} \label{exe4.16:2a} \\
        & \hoare{\subst{Q_2}{v}{v_2}}{v_2 + 5}{v.v = v_1 * Q_1},
        \tag{\ref{exe4.16:2}b} \label{exe4.16:2b}
      \end{align}

      Then by \rulenamestyle{Ht-load}, \eqref{exe4.16:2a} holds for
      $v_2 = m$ and $Q_2 := \loc \gmapsto m$, and \eqref{exe4.16:2b}
      becomes,
      \begin{align*}
        \hoare{\loc \gmapsto m}{m + 5}{v.v = v_1 * Q_1}.
      \end{align*}
      By \rulenamestyle{Ht-op}, it holds for $v_1 = m + 5$ and $Q_1 := 
      \loc \gmapsto m$. Then \eqref{exe4.16:3} becomes
      \begin{align*}
        & \hoare{\loc \gmapsto m}{\mapelem{\loc}{m + 5}}
        {v.v = () * \loc \gmapsto (m + 5)} 
      \end{align*}
      which is obvious by \rulenamestyle{Ht-store}.

    \item Apply \rulenamestyle{Ht-let-det} to the goal
      \begin{align}
        \hoare{P}{\subst{e}{x}{m + 5}}{v.Q} \proves &
        \hoare{P * \loc \gmapsto m}{\deref{\loc} + 5}{x.x = v \wedge R}
        \label{exe4.16:4} \\
        \hoare{P}{\subst{e}{x}{m + 5}}{v.Q} \proves &
        \hoare{\subst{R}{x}{v}}{\subst{e}{x}{v}}{v.Q}
        \label{exe4.16:5}
      \end{align}

      Similar to \ref{exe4.16.1}, \eqref{exe4.16:4} is immediate for
      $v = m + 5$ and $R := \loc \gmapsto m * P$. Then \eqref{exe4.16:5}
      becomes
      \begin{align*}
        \hoare{P}{\subst{e}{x}{m + 5}}{v.Q} \proves &
        \hoare{\loc \gmapsto m * P}{\subst{e}{x}{m + 5}}{v.Q}
      \end{align*}
      which is immediate by apply \rulenamestyle{$*$-weak} and
      \rulenamestyle{Asm}.

    \item Apply \rulenamestyle{Ht-let-det} to the goal,
      \begin{align}
        \hoare{P}{\subst{e}{x}{v_1}}{v.Q} \proves &
        \hoare{u = (v_1, v_2) * P}{\pi_1 u}{x.x = x_1 \wedge R} \label{exe4.16:6} \\
        \hoare{P}{\subst{e}{x}{v_1}}{v.Q} \proves &
        \hoare{\subst{R}{x}{x_1}}{\subst{e}{x}{x_1}}{v.Q} \label{exe4.16:7}
      \end{align}
      Substitute $u$ in \eqref{exe4.16:6},
      \begin{align*}
        \hoare{P}{\pi_1 (v_1, v_2)}{x.x = x_1 \wedge R}.
      \end{align*}
      It is obvious by \rulenamestyle{Ht-proj} for $x_1 = v_1$ and $R := P$.
      Then \eqref{exe4.16:7} becomes
      \begin{align*}
        \hoare{P}{\subst{e}{x}{v_1}}{v.Q} \proves &
        \hoare{P}{\subst{e}{x}{v_1}}{v.Q}
      \end{align*}
      which is immediate by \rulenamestyle{Asm}.
  \end{enumerate}
\end{proof}
