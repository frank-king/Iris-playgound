\section{The Logics of Resources}

\setcounter{subsection}{1}
\subsection{Rules for separating conjunction and magic wand}

\setcounter{exercise}{1}
\begin{exercise}
Following the example above derive the following two rules:
\begin{mathpar}
  \infer
  {x \not\in \mathrm{FV}(P)}
  {P * \exists x. \Phi \provesIff \exists x. P * \Phi}
  \and
  \infer
  {x \not\in \mathrm{FV}(P)}
  {P \wedge \exists x. \Phi \provesIff \exists x. P \wedge \Phi}
\end{mathpar}
where $\mathrm{FV}$ stands for \textit{free variables}.
\end{exercise}

\begin{proof}
Let's prove these two rules one by one, and direction by direction.

\begin{enumerate}
\item The direction from left to right:
  \begin{mathpar}
    \infern{$\wand$E} {
      \inferN{$\exists$E} {
        \inferN{$\wand$I} {
          \inferN{$\exists$I} {
            \vctx, x: \type \mid P * \Phi \proves P * \Phi \and
            \vctx, x: \type \proves \wtt{x}{\type} \and
            x \not\in \mathrm{FV}(P)
          } {\vctx, x: \type \mid P * \Phi \proves \exists x: \type. P * \Phi}
        } {\vctx, x: \type \mid \Phi \proves P \wand \exists x: \type. P * \Phi}
      } {
        \vctx\mid\exists x: \type. \Phi \proves
        P \wand \exists x: \type. P * \Phi
      }
    } {\vctx\mid P * \exists x: \type. \Phi \proves \exists x: \type. P * \Phi}.
  \end{mathpar}

  The direction from right to left:
  \begin{mathpar}
    \infern{$\exists$E} {
      \inferN{$*$-mono} {
        \inferN{$\exists$I} {
          \vctx, x: \type \mid \Phi \proves \Phi \and
          \vctx, x: \type \proves \wtt{x}{\type}
        } {\vctx, x: \type \mid \Phi \proves \exists x: \type. \Phi}
      } {\vctx, x: \type \mid P * \Phi \proves P * \exists x: \type. \Phi}
    } {\vctx\mid \exists x: \type. P * \Phi \proves P * \exists x: \type. \Phi}.
  \end{mathpar}
\item To prove the direction from left to right, we first prove a lemma
  \begin{mathpar}
    \infer[$\Ra$E$'$]
    {\propB \proves \prop \Ra \propC}
    {\prop \wedge \propB \proves \propC}.
  \end{mathpar}
  It can be derived by
  \begin{mathpar}
    \infern{$\Ra$E} {
      \infern{Cut} {
        \infern{$\wedge$ER}
        {\inferN{Asm} { } {\prop \wedge \propB \proves \prop \wedge \propB}}
        {\prop \wedge \propB \proves \propB} \and
        \propB \proves \prop \Ra \propC
      } {\prop \wedge \propB \proves \prop \Ra \propC}
      \and
      \inferN{$\wedge$EL}
      {\inferN{Asm} { } {\prop \wedge \propB \proves \prop \wedge \propB}}
      {\prop \wedge \propB \proves \prop}
    } {\prop \wedge \propB \proves \propC}.
  \end{mathpar}

  Then the original rule can be derived as
  \begin{mathpar}
    \infern{$\Ra$E$'$} {
      \inferN{$\exists$E} {
        \inferN{$\Ra$I} {
          \inferN{$\exists$I} {
            \vctx, x: \type \mid P \wedge \Phi \proves P \wedge \Phi \and
            \vctx, x: \type \proves \wtt{x}{\type} \and
            x \not\in \mathrm{FV}(P)
          } {
            \vctx, x: \type \mid P \wedge \Phi \proves
            \exists x: \type. P \wedge \Phi
          }
        } {
          \vctx, x: \type \mid \Phi \proves
          P \Ra \exists x: \type. P \wedge \Phi
        }
      } {
        \vctx\mid \exists x: \type. \Phi \proves
        P \Ra \exists x: \type. P \wedge \Phi
      }
    } {
      \vctx\mid P \wedge \exists x: \type. \Phi \proves
      \exists x: \type. P \wedge \Phi
    }.
  \end{mathpar}

  The direction from right to left:
  \begin{mathpar}
    \infern{$\exists$E} {
      \inferN{$\wedge$I} {
        \infern{$\wedge$EL}{}{\vctx, x: \type \mid P \wedge \Phi \proves P} \and
        \inferN{$\exists$I} {
          \infern{$\wedge$ER}{}{\vctx, x: \type \mid P \wedge \Phi \proves \Phi}
          \and
          \vctx, x: \type \proves \wtt{x}{\type}
        } {\vctx, x: \type \mid P \wedge \Phi \proves\exists x: \type. \Phi}
      } {\vctx, x: \type \mid P \wedge \Phi \proves P \wedge \exists x: \type. \Phi}
    } {
      \vctx\mid \exists x: \type. P \wedge \Phi \proves
      P \wedge \exists x: \type. \Phi
    }.
  \end{mathpar}

\end{enumerate}
\end{proof}

