\section{Separation Logic for Sequential Programs}
\begin{exercise}
The rule
\begin{mathpar}
  \infer[Ht-frame-invalid]
  {S \proves \hoare{P}{e}{v.Q}}
  {S \proves \hoare{P \wedge R}{e}{v.Q \wedge R}}

\end{mathpar}
is not sound.

Come up with a counterexample to the above rule.
\end{exercise}

\begin{proof}
  We have \[
    S \proves 
    \hoare{x \gmapsto 0 \wedge y \gmapsto 1}
    {\mapelem{y}{\deref{x}}}
    {v.v = \TT \wedge y \gmapsto 0},
  \] but not \[
    S \proves 
    \hoare{x \gmapsto 0 \wedge y \gmapsto 1 \wedge y \gmapsto 1}
    {\mapelem{y}{\deref{x}}}
    {v.v = \TT \wedge y \mapsto 0 \wedge y \gmapsto 1}.
  \]
\end{proof}

\begin{exercise}
  Prove \[
    S \proves 
    (\hoare{P}{e}{v.Q} \Ra \forall R: \Prop, \hoare{P * R}{e}{v.Q * R}).
  \]
\end{exercise}

\begin{proof}
  It can be derived by
  \begin{mathpar}
    \infern{$\Ra$I} {
      \inferN{$\forall$I} {
        \inferN{Ht-frame} {
          \inferN{$\wedge$ER}{ }{
            \vctx, R: \Prop \mid S \wedge \hoare{P}{e}{v.Q} \proves
            \hoare{P}{e}{v.Q}
          }
        } {
          \vctx, R: \Prop \mid S \wedge \hoare{P}{e}{v.Q} \proves
          \hoare{P * R}{e}{v.Q * R}
        }
      } {
        \vctx\mid S \wedge \hoare{P}{e}{v.Q} \proves
        \forall R: \Prop, \hoare{P * R}{e}{v.Q * R}
      }
    } {
      \vctx\mid S \proves 
      (\hoare{P}{e}{v.Q} \Ra \forall R: \Prop, \hoare{P * R}{e}{v.Q * R})
    }.
  \end{mathpar}
\end{proof}

\begin{exercise}
  Use \rulenamestyle{Ht-bind} to show $\hoare{\TRUE}{3 + 4 + 5}{v.v = 12}$.
\end{exercise}

\begin{proof}
  It is obvious that the primary evaluation context of $3 + 4 + 5$ is $3 + 4$.
  We must show $\hoare{\TRUE}{3 + 4}{w.Q}$ and 
  $\forall w. \hoare{Q}{w + 5}{w.w = 12}$ for some $Q$.

  Let $Q := w = 7$, then these two proposition is immediate to prove.
\end{proof}

\begin{exercise}
 Prove the derived rule $P * Q \proves P \wedge Q$ for any propositions $P$ and $Q$.
\end{exercise}

\begin{proof}
  It can be derived by
  \begin{mathpar}
    \infern{$\wedge$I} {
      \infern{$*$-weak}{ }{P * Q \proves P} \and
      \inferN{$*$-weak, $*$-comm}{ }{P * Q \proves Q}
    } {P * Q \proves P \wedge Q}.
  \end{mathpar}
\end{proof}

\begin{exercise}
  Use the intuitive reading of Hoare triples to explain why the above rules are 
  sound.
  \begin{mathpar}
    \inferB[Ht-disj] {
      S \proves \hoare{P}{e}{v.R} \and S \proves \hoare{Q}{e}{v.R}
    } {S \proves \hoare{P \vee Q}{e}{v.R}}
    \and
    \inferB[Ht-exist] {
      x \not\in Q \and S \proves \forall x. \hoare{P}{e}{v.Q}
    } {
      x \not\in Q \and S \proves \hoare{\exists x.P}{e}{v.Q}
    }
  \end{mathpar}
\end{exercise}

% \begin{proof}
  \hspace*{-2em} \textit{Proof.}
  Firstly, consider rule \rulenamestyle{Ht-disj}, the direction from top to 
  bottom.

  The hypothses tell us, that for any precondition $P$, after execution of $e$,
  then $v.R$ holds, and so for any precondition $Q$.
  It is reasonable that for any precondition $P \vee Q$, after execution of
  $e$, then $v.R$ holds.

  Then the opposite direction.

  The hypothsis tells us, that for any precondition $P \vee Q$, after execution
  of $e$, then $v.R$ holds.
  In the precondition, $P \vee Q$ (i.e., either $P$ or $Q$ holds) proves
  the postcondition $v.R$, then separately, both of $P$ and $Q$ as a precondition
  in alone, should prove the postcondition $v.R$.
  
  Secondly, consider rule \rulenamestyle{Ht-exist}, the direction from top to 
  bottom.

  The hypothsis tells us, that for all $x$, for any precondition $P$, after
  execution of $e$, then $v.Q$ holds.
  We peek any $x$ where $P$ holds, then for this particular $x$, the precondition
  $P$ must prove the postcondition $v.Q$. So $\exists x. P$, as a precondition,
  should prove the same postcondition, too.

  Then the opposite direction.

  \textit{Admitted.} (Need more strict definition of Hoare triple).
  % TODO
  % The hypothsis tells us, that for any precondition $\exists x. P$, after execution
  % of $e$, then $v.Q$ holds.
  % The precondition $\exists x. P$ means all the $x$ such that $P$ holds. So for all
  % $x$, the precondition $P$ should prove the postcondition $v.Q$.
% \end{proof}

\setcounter{exercise}{7}
\begin{exercise}
  Prove the following derived rule. For any \textit{value} $u$ and expression $e$
  we have
  \begin{mathpar}
    \infer{S \proves \hoare{P}{e}{v.Q}}
    {S \proves \hoare{P}{\pi_1(e, u)}{v.Q}}
  \end{mathpar}
  It is important that the second component is a value. Show that the following
  rule is not valid in general if $e_1$ is allowed to be an arbitrary expression.
  \begin{mathpar}
    \infer{S \proves \hoare{P}{e}{v.Q}}
    {S \proves \hoare{P}{\pi_1(e, e_1)}{v.Q}}
  \end{mathpar}
  Hint: What if $e$ and $e_1$ read or write to the same location?

  The problem is that we know nothing about the behaviour of $e_1$. But we can
  specify its behaviour using Hoare triples. Come up with some propositions 
  $P_1$ and $P_2$ or some conditions on $P_1$ and $P_2$ such that the following 
  rule
  \begin{mathpar}
    \infer{
      S \proves \hoare{P}{e}{v.Q} \and
      S \proves \hoare{P_1}{e_1}{v.P_2}
    } {S \proves \hoare{P}{\pi_1(e, e_1)}{v.Q}}
  \end{mathpar}
  is derivable.
\end{exercise}

\begin{proof}
  \begin{enumerate}
  \item First, we prove the derivation by \rulenamestyle{Ht-bind}.
  \begin{mathpar}
    \infer{S \proves \hoare{P}{e}{v.Q}}
    {S \proves \hoare{P}{\pi_1(e, u)}{v.Q}}
  \end{mathpar}
  
  Applying \rulenamestyle{Ht-bind} to the goal $S \proves \hoare{P}{\pi_1(e, u)}
  {v.Q}$, we have to show 
  \begin{align}
    S \proves & \hoare{P}{(e, u)}{v_1.Q_1}, \label{exe4.8:1} \\
    S \proves & \hoare{\subst{Q_1}{v_1}{u_1}}{\pi_1 u_1}{v.Q}, \label{exe4.8:2}
  \end{align}
  for some $Q_1$.

  Applying \rulenamestyle{Ht-bind} again to \eqref{exe4.8:1}, we have to show
  \begin{align}
    S \proves & \hoare{P}{e}{v_2.Q_2}, \label{exe4.8:3} \\
    S \proves & \hoare{\subst{Q_2}{v_2}{u_2}}{(u_2, u)}{v_1.Q_1}, \label{exe4.8:4}
  \end{align}
  for some $Q_2$.

  From the hypothesis $S \proves \hoare{P}{e}{v.Q}$, \eqref{exe4.8:3} is immediate 
  for $Q_2 := \subst{Q}{v}{v_2}$. Then \eqref{exe4.8:4} becomes
  \begin{align}
    S \proves & \hoare{\subst{Q}{v}{u_2}}{(u_2, u)}{v_1.Q_1}.
    \tag{\ref{exe4.8:4}a} \label{exe4.8:4a}
  \end{align}

  Since $(u_2, u)$ is a value, we have
  \begin{align*}
    S \proves & \hoare{\TRUE}{(u_2, u)}{v_1.v_1 = (u_2, u)},
  \end{align*}
  by \rulenamestyle{Ht-ret}, which derives 
  \begin{align}
    S \proves & 
    \hoare{\TRUE * \subst{Q}{v}{u_2}}{(u_2, u)}
    {v_1.v_1 = (u_2, u) * \subst{Q}{v}{u_2}},
    \tag{\ref{exe4.8:4}b} \label{exe4.8:4b}
  \end{align}
  by \rulenamestyle{Ht-frame}. Hence, \eqref{exe4.8:4a} is reached for $Q_1 :=
  v_1 = (u_2, u) * \subst{Q}{v}{u_2}$.

  Now we only have to show the remained \eqref{exe4.8:2}, which becomes
  \begin{align}
    S \proves & \hoare{u_1 = (u_2, u) * \subst{Q}{v}{u_2}}{\pi_1 u_1}{v.Q}.
    \tag{\ref{exe4.8:2}a} \label{exe:4.8:2a}
  \end{align}

  The persistent proposition $u_1 = (u_2, u)$ can be moved out of the Hoare
  precondition, and substitute the Hoare triple with it, we have
  \begin{align}
    S \proves & \hoare{\subst{Q}{v}{u_2}}{\pi_1 (u_2, u)}{v.Q}.
    \tag{\ref{exe4.8:2}b} \label{exe:4.8:2b}
  \end{align}

  Applying \rulenamestyle{Ht-proj}, we have
  \begin{align*}
    S \proves & \hoare{\TRUE}{\pi_1 (u_2, u)}{v.v = u_2}.
  \end{align*}
  And then by \rulenamestyle{Ht-frame} again, there is
  \begin{align*}
    S \proves &
    \hoare{\TRUE * \subst{Q}{v}{u_2}}{\pi_1 (u_2, u)}
    {v.v = u_2 * \subst{Q}{v}{u_2}}.
  \end{align*}
  Apparently, \eqref{exe:4.8:2b} is derivable using \rulenamestyle{Ht-csq}.

  \item Now we show that rule 
  \begin{mathpar}
    \infer{S \proves \hoare{P}{e}{v.Q}}
    {S \proves \hoare{P}{\pi_1(e, e_1)}{v.Q}}
  \end{mathpar}
  is invalid by a counterexample.

  Let $P := \loc \gmapsto 0$, $e := \mapelem{\loc}{0}$, $e_1 := 
  \mapelem{\loc}{1}$, and $Q := v = () \wedge \loc \gmapsto 0$.

  It is obvious that $\hoare{P}{e}{v.Q}$ holds but $\hoare{P}{\pi_1(e, e_1)}{v.Q}$
  not, since \[
    \hoare{\loc \gmapsto 0}{\pi_1(\mapelem{\loc}{0}, \mapelem{\loc}{1})}
    {v.v = () \wedge \loc \gmapsto 1}.
  \]

  \item Finally, we add some restrictions on $e_1$ to let
  \begin{mathpar}
    \infer{
      S \proves \hoare{P}{e}{v.Q} \and
      S \proves \hoare{P_1}{e_1}{v.P_2}
    } {S \proves \hoare{P}{\pi_1(e, e_1)}{v.Q}}
  \end{mathpar}
  hold.

  The idea is that, the expression $(e, e_1)$ is evaluated sequentially,
  first $e$ then $e_1$, and $\pi_1(e, e_1)$ selects the evaluation result of 
  $e$, so the evaluation result of $e_1$ doesn't matter, and it can refer 
  to the resource claimed by $Q$ but it must not mutate it.

  Such $P_1$ and $P_2$ can be $P_1 := \exists v. Q$ and $P_2 := Q$.

  First, apply \rulenamestyle{Ht-bind} to the goal $S \proves \hoare{P}
  {\pi_1(e, e_1)}{v.Q}$, 
  \begin{align}
    S \proves & \hoare{\subst{Q}{v}{u_2}}{(u_2, e_1)}{v_1.Q_1}, \label{exe4.8:5} \\
    S \proves & \hoare{\subst{Q_1}{v_1}{u_1}}{\pi_1 u_1}{v.Q}. \label{exe4.8:6}
  \end{align}
  
  Then apply \rulenamestyle{Ht-bind} to \eqref{exe4.8:5},
  \begin{align}
    S \proves & \hoare{\subst{Q}{v}{u_2}}{e_1}{w_1.R_1}, \label{exe4.8:7} \\
    S \proves & \hoare{\subst{R_1}{w_1}{w_2}}{(u_2, w_2)}{v_1.Q_1}. \label{exe4.8:8}
  \end{align}

  Applying \rulenamestyle{Ht-exist} and \rulenamestyle{$\forall$-E} to the
  second hypothesis, there is
  \begin{align*}
    S \proves & \hoare{Q}{e_1}{v.Q}.
  \end{align*}

  Let 
  \begin{align*}
    Q_1 := & v = u_2 * v_1 = (u_2, w_2) * \subst{Q}{v}{u_2}, \\
    R_1 := & w_1 = u_2 * \subst{Q}{v}{u_2},
  \end{align*}
  then \eqref{exe4.8:6}, \eqref{exe4.8:7} and \eqref{exe4.8:8} become
  \begin{align}
    S \proves & 
    \hoare{v = u_2 * u_1 = (u_2, w_2) * \subst{Q}{v}{u_2}}{\pi_1 u_1}{v.Q},
    \tag{\ref{exe4.8:6}a} \label{exe4.8:6a} \\
    %
    S \proves & \hoare{\subst{Q}{v}{u_2}}{e_1}{w_1.w_1 = u_2 * \subst{Q}{v}{u_2}},
    \tag{\ref{exe4.8:7}a} \label{exe4.8:7a} \\
    %
    S \proves & \hoare{w_2 = u_2 * \subst{Q}{v}{u_2}}{(u_2, w_2)\notag \\&}
    {v_1.v = u_2 * v_1 = (u_2, w_2) * \subst{Q}{v}{u_2}}.
    \tag{\ref{exe4.8:8}a} \label{exe4.8:8a}
  \end{align}
  
  \eqref{exe4.8:7a} is immediate from the second hypothesis.

  Substitute \eqref{exe4.8:6a} with $u_1 = (u_2, w_2)$ and $v = u_2$, and
  substitute \eqref{exe4.8:8a} with $w_2 = u_2$,
  \begin{align}
    S \proves & 
    \hoare{Q}{\pi_1 (v, w_2)}{v.Q}, \tag{\ref{exe4.8:6}b} \label{exe4.8:6b} \\
    S \proves & \hoare{\subst{Q}{v}{u_2}}{(u_2, u_2)}
    {v_1.v = u_2 * v_1 = (u_2, u_2) * \subst{Q}{v}{u_2}}.
    \tag{\ref{exe4.8:8}b} \label{exe4.8:8b}
  \end{align}

  \eqref{exe4.8:6b} is immediate by \rulenamestyle{Ht-proj}, and 
  \eqref{exe4.8:8b} is immediate by \rulenamestyle{Ht-ret} and 
  \rulenamestyle{Ht-frame}.
  \end{enumerate}
\end{proof}

\begin{exercise}
  From \rulenamestyle{Ht-If} we can derive two, perhaps more natural, rules,
  which are simpler to use. They require us to only prove a specification of
  the branch which will be taken.
  \begin{mathpar}
    \infer[Ht-If-True] {
      \hoare{P * v = \True}{e_2}{u.Q}
    } {
      \hoare{P * v = \True}{\If v then e_2 \Else e_3}{u.Q}
    } \and
    \infer[Ht-If-False] {
      \hoare{P * v = \False}{e_3}{u.Q}
    } {
      \hoare{P * v = \False}{\If v then e_2 \Else e_3}{u.Q}
    }
  \end{mathpar}
  Derive \rulenamestyle{Ht-If-True} and \rulenamestyle{Ht-If-False} from 
  \rulenamestyle{Ht-If}.
\end{exercise}

\begin{proof}
  \begin{mathpar}
    \infern{Ht-If} {
      \infer{\hoare{P * v = \True}{e_2}{u.Q}}
      {\hoare{P * v = \True * v = \True}{e_2}{u.Q}} \and
      \infer{
        \inferN{Ht-false}{ }{\hoare{\FALSE}{e_3}{u.Q}}
      } {\hoare{P * v = \True * v = \False}{e_3}{u.Q}}
    } {
      \hoare{P * v = \True}{\If v then e_2 \Else e_3}{u.Q}
    } \and
    \infern{Ht-If} {
      \infer{
        \inferN{Ht-false}{ }{\hoare{\FALSE}{e_2}{u.Q}}
      } {\hoare{P * v = \False * v = \True}{e_2}{u.Q}} \and
      \infer{\hoare{P * v = \False}{e_3}{u.Q}}
      {\hoare{P * v = \False * v = \False}{e_3}{u.Q}}
    } {
      \hoare{P * v = \False}{\If v then e_2 \Else e_3}{u.Q}
    }
  \end{mathpar}
\end{proof}

\begin{exercise}
  Show the following derived rule for any expression $e$ and $i \in \{1, 2\}$.
  \begin{mathpar}
    \infer {
      S \proves \hoare{P}{e}{v.v = \textlog{inj}_i u * Q} \and
      S \proves \hoare{\subst{Q}{v}{\textlog{inj}_i u}}{\subst{e_i}{x_i}{u}}{v.R}
    } {
      S \proves \hoare{P}
      {\Match e with \textlog{inj}_1 x_1 => e_1 | \textlog{inj}_2 x_2 => e_2 end}
      {v.R}
    }
  \end{mathpar}
\end{exercise}

\begin{proof}
  It can be derived by \rulenamestyle{Ht-bind} and \rulenamestyle{Ht-Match}.
\end{proof}

\subsection{Derived rules for Hoare triples}
\begin{exercise}
  Show the following derived rule
  \begin{mathpar}
    \infer[Ht-Pre-Eq] {
      \vctx\mid S \proves 
      \hoare{\subst{P}{x}{v}}{\subst{e}{x}{v}}{u.\subst{Q}{x}{v}}
    } {\vctx, x: \Val \mid S \proves \hoare{x = v \wedge P}{e}{u.Q}}
  \end{mathpar}
\end{exercise}

\begin{proof}
  \begin{mathpar}
    \infern{Ht-Eq} {
      \inferN{Eq} {
        \inferrule*{
          \vctx\mid S \proves 
          \hoare{\subst{P}{x}{v}}{\subst{e}{x}{v}}{u.\subst{Q}{x}{v}}
        } {
          \vctx, x: \Val \mid S \wedge x = v \proves 
          \hoare{\subst{P}{x}{v}}{\subst{e}{x}{v}}{u.\subst{Q}{x}{v}}
        } \and
        S \wedge x = v \proves x = v
      } {\vctx, x: \Val \mid S \wedge x = v \proves \hoare{P}{e}{u.Q}}
    } {\vctx, x: \Val \mid S \proves \hoare{x = v \wedge P}{e}{u.Q}}
  \end{mathpar}
\end{proof}

\begin{exercise}
  Recall we define the let expression $\Let x = e1 in e2$ using abstraction and 
  application. Show the following derived rule
  \begin{mathpar}
    \infer[Ht-let] {
      S \proves \hoare{P}{e_1}{x.Q} \and
      S \proves \forall v. \hoare{\subst{Q}{x}{v}}{\subst{e_2}{x}{v}}{u.R}
    } {S \proves \hoare{P}{\Let x = e_1 in e_2}{u.R}}
  \end{mathpar}
  Using this rule is perhaps a bit inconvenient since most of the time the result
  of evaluating $e_1$ will be a single value and the postcondition $Q$ will be of
  the form $x = v \wedge Q'$ for some value $v$.

  The following rule, a special case of the above rule reflects this common case.
  Derive the rule from the rule \rulenamestyle{Ht-let}.
  \begin{mathpar}
    \infer[Ht-let-det] {
      S \proves \hoare{P}{e_1}{x.x = v \wedge Q} \and
      S \proves \hoare{\subst{Q}{x}{v}}{\subst{e_2}{x}{v}}{u.R}
    } {S \proves \hoare{P}{\Let x = e_1 in e_2}{u.R}}
  \end{mathpar}
  Define the sequencing expression $e_1; e_2$ such that when this expression is
  run first $e_1$ is evaluated to a value, the value is discarded, and then $e_2$
  is evaluated. Show the following specifications for the defined construct.
  \begin{mathpar}
    \infer[Ht-seq] {
      S \proves \hoare{P}{e_1}{v.Q} \and
      S \proves \hoare{\exists x. Q}{e_2}{u.R}
    } {
      S \proves \hoare{P}{e_1; e_2}{u.R}
    } \and
    \infer {
      S \proves \hoare{P}{e_1}{\_.Q} \and
      S \proves \hoare{Q}{e_2}{u.R}
    } {
      S \proves \hoare{P}{e_1 ; e_2}{u.R}
    }
  \end{mathpar}
  where $\_.Q$ means that $Q$ does not mention the return value.
\end{exercise}

\begin{proof}
  \begin{enumerate}
    \item Notice that $\Let x = e_1 in e_2$ is desugared to $(\Rec f(x)=e_2) e_1$,
      then \rulenamestyle{Ht-let} becomes
      \begin{mathpar}
        \infer {
          S \proves \hoare{P}{e_1}{x.Q} \and
          S \proves \forall v. \hoare{\subst{Q}{x}{v}}{\subst{e_2}{x}{v}}{u.R}
        } {S \proves \hoare{P}{(\Rec f(x)=e_2) e_1}{u.R}}
      \end{mathpar}

      Applying \rulenamestyle{Ht-bind} to $S \proves 
      \hoare{P}{(\Rec f(x)=e_2) e_1}{u.R}$, we have to show
      \begin{align}
        S \proves & \hoare{P}{e_1}{x.Q}, \label{exe4.12:1} \\
        S \proves & \forall v. \hoare{\subst{Q}{v}{x}}{(\Rec f(x)=e_2) v}{u.R},
        \label{exe4.12:2}
      \end{align}
      where the original $x$ \eqref{exe4.12:2} has been substituted by $v$ in
      order not to be confused with the $x$ in $f(x)$.

      \eqref{exe4.12:1} is immediate from the first hypothsis. Applying
      \rulenamestyle{Ht-rec} to \eqref{exe4.12:2}, we have to show
      \begin{align*}
        \vctx, g: \Val \mid S \wedge \forall v. \hoare{Q}{g\, v}{u.R} \proves
        \forall v. \hoare{\subst{Q}{x}{v}}{\subst{\subst{e_2}{f}{g}}{x}{v}}{u.R}
      \end{align*}

      Since $\Rec f(x)=e_2$ is not recursive, $e_2$ does not contain $f$,
      we can omit the $g$, then
      \begin{align*}
        \vctx, g: \Val \mid S \wedge \forall v. \hoare{Q}{g\, v}{u.R} \proves
        \forall v. \hoare{\subst{Q}{x}{v}}{\subst{e_2}{x}{v}}{u.R}
      \end{align*}
      which is immediate from the second hypothsis.
      
    \item Notice that $e_1; e_2$ is desugared to $\Let \_ = e_1 in e_2$, then
      \rulenamestyle{Ht-seq} is immediate from \rulenamestyle{Ht-let-det}.
  \end{enumerate}
\end{proof}

\begin{exercise}
  Recall we defined $\Lam x.e$ to be $\Rec f(x)=e$ where $f$ is some fresh variable.
  Show the following derived rule.
  \begin{mathpar}
    \infer[Ht-beta] 
    {S \proves \hoare{P}{\subst{e}{x}{v}}{u.Q}}
    {S \proves \hoare{P}{(\Lam x.e)v}{u.Q}}
  \end{mathpar}
\end{exercise}

\begin{proof}
  Desugar $(\Lam x.e) v$ to $(\Rec f(x)=e) v$, and apply \rulenamestyle{Ht-rec}
  to it, then we have to show
  \begin{align*}
    \vctx, g: \Val \mid S \wedge \forall v. \hoare{P}{g\, v}{u.Q} \proves
    \hoare{P}{\subst{e}{x}{v}}{u.Q}
  \end{align*}
  which is immediate from the hypothesis.
\end{proof}

\begin{exercise}
  Derive the following rule.
  \begin{mathpar}
    \infer[Ht-bind-det] {
      E \text{is an eval. context} \and
      S \proves \hoare{P}{e}{x.x = u \wedge Q} \and
      S \proves \hoare{\subst{Q}{x}{u}}{E[u]}{w. R}
    } {S \proves \hoare{P}{E[e]}{w. R}}
  \end{mathpar}
\end{exercise}

\begin{proof}
  Apply \rulenamestyle{Ht-bind} with $Q := x = u \wedge Q$ to the goal, then we
  have to show
  \begin{align*}
    S \proves \forall v. \hoare{u = v \wedge \subst{Q}{x}{v}}{E[v]}{w. R}
  \end{align*}
  
  Extract $u = v$ with \rulenamestyle{Ht-eq} and substitute $v$ with $u$ in the
  above entailment,
  \begin{align*}
    S \wedge u = u \proves \hoare{\subst{Q}{x}{u}}{E[u]}{w. R}
  \end{align*}
  which is immediate from the hypothesis.
\end{proof}

\setcounter{exercise}{15}
\begin{exercise}
  Show the following triples and entailments in detail.
  \begin{enumerate}
    \item \[ \hoare{R * \loc \gmapsto m}{\mapelem{\loc}{\deref{\loc} + 5}}
      {v.R * v = () * \loc \gmapsto (m + 5)} \]
    \item \[ \hoare{P}{\subst{e}{x}{m + 5}}{v.Q} \proves
      \hoare{P * \loc \gmapsto m}{\Let x = \deref{\loc} + 5 in e}
      {v.Q * \loc \gmapsto m} \]
    \item Assuming $u$ does not appear in $P$ and $Q$ show the following
      entailment.
      \[ \hoare{P}{\subst{e}{x}{v_1}}{v.Q} \proves
      \hoare{u = (v_1, v_2) * P}{\Let x = \pi_1 u in e}{v.Q} \]
  \end{enumerate}
\end{exercise}

\begin{proof}
  \begin{enumerate}
    \item \label{exe4.16.1} 
      Apply \rulenamestyle{Ht-bind-det} to the goal
      \begin{align}
        \hoare{\loc \gmapsto m}{\mapelem{\loc}{\deref{\loc} + 5}}
        {v.v = () * \loc \gmapsto (m + 5)} \label{exe4.16:1}
      \end{align}
      
      Apply \rulenamestyle{Ht-bind-det} to \eqref{exe4.16:1},
      \begin{align}
        & \hoare{\loc \gmapsto m}{\deref{\loc} + 5}{v.v = v_1 * Q_1},
        \label{exe4.16:2} \\
        & \hoare{\subst{Q_1}{v}{v_1}}{\mapelem{\loc}{v_1}}
        {v.v = () * \loc \gmapsto (m + 5)} \label{exe4.16:3}
      \end{align}

      Apply \rulenamestyle{Ht-bind-det} to \eqref{exe4.16:2},
      \begin{align}
        & \hoare{\loc \gmapsto m}{\deref{\loc}}{v.v = v_2 * Q_2},
        \tag{\ref{exe4.16:2}a} \label{exe4.16:2a} \\
        & \hoare{\subst{Q_2}{v}{v_2}}{v_2 + 5}{v.v = v_1 * Q_1},
        \tag{\ref{exe4.16:2}b} \label{exe4.16:2b}
      \end{align}

      Then by \rulenamestyle{Ht-load}, \eqref{exe4.16:2a} holds for
      $v_2 = m$ and $Q_2 := \loc \gmapsto m$, and \eqref{exe4.16:2b}
      becomes,
      \begin{align*}
        \hoare{\loc \gmapsto m}{m + 5}{v.v = v_1 * Q_1}.
      \end{align*}
      By \rulenamestyle{Ht-op}, it holds for $v_1 = m + 5$ and $Q_1 := 
      \loc \gmapsto m$. Then \eqref{exe4.16:3} becomes
      \begin{align*}
        & \hoare{\loc \gmapsto m}{\mapelem{\loc}{m + 5}}
        {v.v = () * \loc \gmapsto (m + 5)} 
      \end{align*}
      which is obvious by \rulenamestyle{Ht-store}.

    \item Apply \rulenamestyle{Ht-let-det} to the goal
      \begin{align}
        \hoare{P}{\subst{e}{x}{m + 5}}{v.Q} \proves &
        \hoare{P * \loc \gmapsto m}{\deref{\loc} + 5}{x.x = v \wedge R}
        \label{exe4.16:4} \\
        \hoare{P}{\subst{e}{x}{m + 5}}{v.Q} \proves &
        \hoare{\subst{R}{x}{v}}{\subst{e}{x}{v}}{v.Q}
        \label{exe4.16:5}
      \end{align}

      Similar to \ref{exe4.16.1}, \eqref{exe4.16:4} is immediate for
      $v = m + 5$ and $R := \loc \gmapsto m * P$. Then \eqref{exe4.16:5}
      becomes
      \begin{align*}
        \hoare{P}{\subst{e}{x}{m + 5}}{v.Q} \proves &
        \hoare{\loc \gmapsto m * P}{\subst{e}{x}{m + 5}}{v.Q}
      \end{align*}
      which is immediate by apply \rulenamestyle{$*$-weak} and
      \rulenamestyle{Asm}.

    \item Apply \rulenamestyle{Ht-let-det} to the goal,
      \begin{align}
        \hoare{P}{\subst{e}{x}{v_1}}{v.Q} \proves &
        \hoare{u = (v_1, v_2) * P}{\pi_1 u}{x.x = x_1 \wedge R} \label{exe4.16:6} \\
        \hoare{P}{\subst{e}{x}{v_1}}{v.Q} \proves &
        \hoare{\subst{R}{x}{x_1}}{\subst{e}{x}{x_1}}{v.Q} \label{exe4.16:7}
      \end{align}
      Substitute $u$ in \eqref{exe4.16:6},
      \begin{align*}
        \hoare{P}{\pi_1 (v_1, v_2)}{x.x = x_1 \wedge R}.
      \end{align*}
      It is obvious by \rulenamestyle{Ht-proj} for $x_1 = v_1$ and $R := P$.
      Then \eqref{exe4.16:7} becomes
      \begin{align*}
        \hoare{P}{\subst{e}{x}{v_1}}{v.Q} \proves &
        \hoare{P}{\subst{e}{x}{v_1}}{v.Q}
      \end{align*}
      which is immediate by \rulenamestyle{Asm}.
  \end{enumerate}
\end{proof}

